\documentclass[12pt,a4paper]{article}

\usepackage[T2A]{fontenc}
\usepackage[utf8]{inputenc}
\usepackage[english,russian]{babel}
\usepackage[left = 1.5cm, right = 1cm, top = 2cm, bottom = 2cm, bindingoffset = 0 cm]{geometry}

\usepackage{amsmath,amsfonts,amssymb,amsthm,mathtools}

\author{Павел Петров}
\title{Лекции по ДГМА}
\date{Семестр 4}
	
\begin{document}
	\maketitle
	\newpage
	\section{Теория поверхностей}
	$\underline{\textbf{Ответственность за то, что написано здесь и далее, полностью с себя снимаю.}}$
	\newline
	$\underline{\textbf{Предыдущие версии файла недействительны.}}$
	\newline
	\newline
	\textbf{Определение 1.} Отображение $f$ области $G$ плоскости на область $\widetilde{G}$ трёхмерного пространства называется \textit{гомеоморфным}, если $f$ взаимно однозначно и взаимно непрерывно. 
	\newline
	\newline
	\textbf{Определение 2.} Множество $\Phi$ точек трёхмерного пространства называется \textit{элементарной поверхностью}, если это множество является образом открытого круга $G$ при гомеоморфном отображении $f$ в пространство. 
	\newline
	\newline
	\textbf{!} $G = \{(x,y)\; |\; x^2 + y^2 < R^2 \}$  \textit{- открытый круг.}
	\newline
	\newline
	\textbf{Определение 3.} Множество $G$ точек плоскости называется \textit{элементарной областью}, если это множество является образом открытого круга $G$ при гомеоморфном отображении $f$ на плоскость. 
	\newline
	\newline
	\textbf{Определение 4.} \textit{Окрестностью точки $M$ множества $\Phi$} называется общая часть множества $\Phi$ и пространственной окрестности $M$.
	\newline
	\newline
	\textbf{Определение 5.} Множество называется \textit{связным}, если любые две его точки можно соединить непрерывной кривой, целиком состоящей из точек этого множества.
	\newline
	\newline
	\textbf{Определение 6.} Множество точек пространства $\Phi$ называется \textit{простой поверхностью}, это множество связно и любая точка этого множества имеет окрестность, являющейся элементарной поверхностью.
	\newline
	\newline
	\textbf{!} \textit{Элементарная повехность является простой повехностью. Обратное неверно. Пример: сфера.}
	\newline
	\newline
	\textbf{Определение 7.} Отображение $f$ простой поверхности $G$ называется \textit{локально-гомеоморфным}, если у каждой точки $G$ есть окрестность, которая гомеоморфно отображается на свой образ.
	\newline
	\newline
	\textbf{Определение 8.} Множество точек пространства $\Phi$ называется \textit{общей поверхностью}, если оно является образом простой поверхности при локально-гомеоморфном отображении.
	\newline
	\newline
	\textit{Замечания к определению 8:}
	\newline
	\textit{1. Окрестность точки общей поверхности - образ окрестности точки на простой поверхности.}
	\newline
	\textit{2. Простая поверхность - это поверхность без самопересечений и без самоналяганий Общая поверхность может иметь их.}
	\newline
	\newline
	\textbf{Определение 9.} Поверхность $\Phi$ называется \textit{регулярной ($k$ раз дифференцируемой)}, если при некотором $k \geq 1$ у каждой точки $\Phi$ есть окрестность, допускающая $k$ раз дифференцируемую параметризацию.
	\newline
	\newline
	То есть окрестность представляет собой гомеоморфное отображение некоторой элементарной области $G$ \textit{(определение элементарной области легко получить, переформулировав определение 2)} в плоскость переменных $(u,v)$ при помощи соотношений \eqref{dif},
	
	\begin{equation}\label{dif}
		x = x(u,v)\quad y = y(u,v)\quad z = z(u,v)
	\end{equation}
	являющимися $k$ раз дифференцируемыми функциями в области $G$.
	Если $k = 1$ и производные непрерывны, то поверхность называется \textit{гладкой}.
	\newline
	\newline
	\textbf{!} Будем говорить, что с помощью соотношений \eqref{dif} в окрестности точки на поверхности вводится \textit{регулярная параметризация} с помощью параметров $u,v$.
	\newline
	\textbf{!} Если вся поверхность $\Phi$ представляет отображение области $G$  при помощи соотношений \eqref{dif}, то говорят, что на $\Phi$ введена \textit{единая параметризация}.
	\newline
	\newline
	\textbf{Определение 10.} Точка регулярной поверхности называется \textit{обыкновенной}, если существует такая регулярная параметризация некоторой её окрестности, что в этой точке
	
	\begin{equation}\label{matr}
	rank
	\begin{pmatrix}
	x'_{u} & y'_{u} & z'_{u}\\
	x'_{v} & y'_{v} & z'_{v}
	\end{pmatrix}
	= 2
	\end{equation}
	\newline
	Если это не так, то точка называется \textit{особой}.
	\newline
	\newline
	\textbf{Определение 11.} $f(u,v) \in C^{k}(G)$, если $f(u,v)\; k$ раз дифференцируема и все её частные производные порядка $k$  непрерывны в $G$.
	\newline
	\newline
	\textbf{Определение 12.} Область $G$ на плоскости будем называть \textit{простой}, если эта область представляет собой простую плоскую поверхность \textit{(то есть $G$ это связная область, каждая точка которой имеет окрестность, являющейся элементарной поверхностью)}.
	\newline
	\newline
	\textit{\textbf{Теорема 13.} Пусть $G$ - простая область плоскости (u,v), $x(u,v),\, y(u,v),\, z(u,v) \in C^{k}(G)$, где $k \geq 1$, и во всех точках области $G$ выполняется условие \eqref{matr}, тогда соотношения \eqref{dif} определяют в пространстве множество $\Phi$, которое представляет собой регулярную, $k$ раз дифференцируемую общую поверхность без особых точек.}
	\newline
	\newline
	\textit{Доказательство.}
	\newline
	Убедимся, что с помощью соотношений \eqref{dif} осуществляется локально-гомеоморфное отображение области $G$ на множество $\Phi$.
	\newline
	Возьмём произвольную точку $M_{0}(x_{0}, y_{0}, z_{0}) \in \Phi$, соответствующую параметрам $(u_{0}, v_{0})$ плоскости $(u,v)$, и зафиксируем её. По условию в каждой точке области $G$ выполняется условие \eqref{matr}, а значит и в точке $(u_{0}, v_{0})$. Для определённости положим, что определитель \eqref{determ} отличен от нуля в $(u_{0}, v_{0})$.
	\newline
	\begin{equation}\label{determ}
		\begin{vmatrix}
			x'_{u} & y'_{u}\\
			x'_{v} & y'_{v}
		\end{vmatrix}
	\end{equation}
	Получили выполнение условий теоремы о неявных функциях, заданных системой уравнений \eqref{syst}:
	\begin{equation}\label{syst}
		\begin{cases}
			x(u,v) - x = 0\\
			y(u,v) - y = 0
		\end{cases}
	\end{equation}
	1. 	\begin{equation}
			\begin{cases}
				x(u_{0},v_{0}) - x_{0} \equiv 0\\
				y(u_{0},v_{0}) - y_{0} \equiv 0
			\end{cases}
		\end{equation}
	2. $x = x(u,v),\, y = y(u,v)$ непрерывны и дифференцируемы.
	\newline
	3. Частные производные непрерывны этих функций.
	\newline
	4. Определитель \eqref{determ}, являющийся якобианом $\frac{D(x,y)}{D(u,v)}$, отличен от нуля в $(u_{0}, v_{0})$.
	\newline
	Тогда найдётся окрестность точки $(x_{0}, y_{0})$ на плоскости $(x,y)$, что в её пределах $\exists ! k$ раз дифференцируемое решение системы \eqref{syst}:
	\begin{equation}
		\begin{cases}\label{revsyst}
			u = u(x,y)\\
			v = v(x,y)
		\end{cases}
	\end{equation}
	\newline
	\newline
	Таким образом мы получили, что некоторая окресность точки $(x_{0}, y_{0})$ представляет собой гомеоморфное отображение некоторой окрестности точки $(u_{0}, v_{0})$ с помощью соотношений $x = x(u,v),\, y = y(u,v)$ \textit{(обратное отображение производится с помощью соотношений $u = u(x,y),\, v = v(x,y)$)}.
	\newline
	Подставим соотношения \eqref{revsyst} в $z = z(u,v)$:
	$z = z(u,v) = z(u(x,y), v(x,y)) = z(x,y)$. Отсюда получаем, что некоторая окрестность точки $M_{0}$ на множестве $\Phi$ является графиком $k$ раз дифференцируемой функции. А это означает, что с помощью функции $z = z(x,y)$ производится гомеоморфное отображение окрестности точки $(x_{0}, y_{0})$ на указанную окрестность точки $M_{0}$ множества $\Phi$. Легко понять, что окрестность точки $(u_{0}, v_{0})$ гомеоморфно отображается на окрестность точки $M_{0}$ множества $\Phi$. 
	\newline 
	Таким образом получили, что у каждой точки простой области $G$ имеется окрестность, которая гомеоморфно отображается на свой образ (окрестность $\Phi$), значит соотношения \eqref{dif} - локально-гомеоморфное отображение $G$ на $\Phi$, следовательно $\Phi$ - общая поверхность. \textit{Теорема доказана.}
	\newline
	\newline
	\textit{Замечание к теореме 13: В процессе доказательства мы установили, что у каждой точки $M_{0}$ поверхности $\Phi$ без особых точек имеется окрестность, однозначно проецирующаяся на одну из координатных плоскостей и являющаяся поэтому графиком $k$ раз дифференцируемой функции. (в доказательства была функция z = z(x,y), но, поменяв какой-нибудь столбец определителя} \eqref{determ} \textit{на столбец состоящий из частных производных функции z = z(x,y), можно получить зависимости y = y(x,z) или x = x(y,z))}
	\newline
	\newline
	Рассмотрим регулярную поверхность $\Phi$. Пусть $\textbf{r}(M)$ - вектор, идущий из начала координат в точку $M$ поверхности. Очевидно, что $\textbf{r}(M)$ - векторная функция переменной точки $M$ поверхности и $\textbf{r}(M)$ - радиус-вектор поверхности $\Phi$.
	\newline
	\newline
	\textbf{Определение 14.} Криволинейная система координат.
	\newline
	Рассмотрим окрестность $M$, являющейся гомеоморфным отображением некоторой элементарной области $G$ при помощи соотношений \eqref{dif}. Очевидно, что \eqref{dif} - координаты переменной точки $M$ поверхности и координаты $\textbf{r}(M)$.
	\newline
	$\textbf{r}(M) = \textbf{r}(x(u,v), y(u,v), z(u,v)) = \textbf{r}(u,v)$ - функция двух переменных.
	\newline
	Зафиксируем $\underline{v}$, тогда конец радиус вектора $\textbf{r}(u,\underline{v})$ описывает в окрестности кривую, называющейся \textit{линией $u$}.
	\newline
	Зафиксируем $\underline{u}$, тогда конец радиус вектора $\textbf{r}(\underline{u},v)$ описывает в окрестности кривую, называющейся \textit{линией $v$}.
	\newline
	Линии $u, v$ называются \textit{координатными линиями} на $\Phi$ в рассматриваемой окрестности. Таким образом, в некоторой окрестности каждой точки поверхности $\Phi$ может быть введена система координатных линий $u, v$, которая называется \textit{криволинейной системой координат} на поверхности.
	\newline
	\newline
	Обратим внимание на определение обыкновенной и особой точки и выясним геометрический смысл $\textbf{r}'_{u}$ и $\textbf{r}'_{v}$. Очевидно, что если выполняется условие \eqref{matr}, то векторы, упомянутые ранее, будут линейно независимыми, в противном случае линейно зависимыми.
	\newline
	Таким образом получили, что если точка $M$ поверхности является обыкновенной, то в окрестности этой точки может быть введена такая параметризация, что $\textbf{r}'_{u}$ и $\textbf{r}'_{v}$ линейно независимы.
	\newline
	\newline
	\textbf{Определение 15.} Плоскость называется \textit{касательной} плоскостью графика функции $z = z(x, y)$ в точке $M_{0}$, если точка $M_{0}$ принадлежит плоскости и вектор секущей к этому графику в точке $M_{0}$ стремится к перпендикуляру нормали этой плоскости при $M \rightarrow M_{0}$.
	\newline
	\newline
	Было доказано, что если $z = z(x, y)$ дифференцируема в $(x_{0}, y_{0})$, то в точке $M_{0} = (x_{0}, y_{0}, z(x_{0}, y_{0}))$ поверхности существует касательная плоскость.
	\newline
	Осталось убедиться в том, что в любой обыкновенной точке гладкой поверхности существует касательная плоскость. Для этого достаточно установить, что некоторая окрестность обыкновенной точки поверхности представляет собой график дифференцируемой функции. Это свойство уже было доказано для любой обыкновенной точки гладкой поверхности \textit{(смотри замечание к теореме 13)}. Следовательно в любой обыкновенной точке гладкой поверхности существует касательная плоскость.
	\newline
	\textit{Замечание: так как векторы $\textbf{r}'_{u}$ и $\textbf{r}'_{v}$ являются касательными к линиям $u, v$, проходящим через $M_{0}$, то эти векторы располагаются в касательной плоскости точки $M_{0}$.}
	\newline
	\newline
	\textbf{Определение 16.} \textit{Нормалью} к поверхности $\Phi$ в точке $M_{0}$ называется прямая, проходящая через $M_{0}$ и перпендикулярная касательной плоскости точки $M_{0}$.
	\newline
	\newline
	\textbf{Определение 17.} \textit{Вектором нормали} к поверхности $\Phi$ в точке $M_{0}$ будем называть ненулевой вектор, коллинеарный нормали в $M_{0}$.
	\newline
	\newline	
	Пусть $M_{0}$ - обыкновенная точка гладкой поверхности $\Phi$ и некоторая окрестность этой точки определена с помощью функции $\textbf{r} = \textbf{r}(u,v)$ такой, что $\textbf{r}'_{u}$ и $\textbf{r}'_{v}$ в $M_{0}$ не коллинеарны. Тогда очевидно, что $\textbf{N} = \left[ \textbf{r}'_{u} , \textbf{r}'_{v} \right]$ является вектором нормали к поверхности. $ \textbf{n} = \frac{\textbf{N}}{|\textbf{N}|} $ - единичный вектор нормали к поверхности.
	\newline
	Поверхность гладкая, следовательно векторные функции $\textbf{N}(u,v)$ и $\textbf{n}(u,v)$ будут непрерывными. Значит в некоторой окрестности каждой точки гладкой поверхности существует \textit{непрерывное векторное поле нормалей}.
	\newline
	\newline
	\textbf{Определение 18.} \textit{Двусторонняя поверхность} - поверхность, на которой в целом существует непрерывное поле нормалей.
	\newline
	\newline
	\textbf{Определение 19.} \textit{Односторонняя поверхность} - поверхность, на которой в целом не существует непрерывного поля нормалей.
	\newline
	\textit{Пример односторонней поверхности - лист Мёбиуса.}
		
	\newpage
	\section{Вспомогательные леммы}
	Начиная с этого момента, рассматриваются только двусторонние поверхности.
	\newline
	\newline
	\textbf{Определение 20.} \textit{Полная поверхность} - поверхность, у которой любая фундаментальная последовательность точек этой поверхности сходится к некоторой точке поверхности.
	\newline
	\textit{Пример полной поверхности: сфера.}
	\newline
	\textit{Пример неполной поверхности: любое открытое связное множество на сфере.}
	\newline
	\newline
	\textbf{Определение 21.} \textit{Часть $\Phi$ имеет размеры меньше $\delta$}, если эта часть помещается внутри некоторой сферы, диаметр которой меньше $\delta$.
	\newline
	\newline
	\textit{\textbf{Лемма 22.}} Пусть $M_{0}$ - обыкновенная точка гладкой поверхности $\Phi$, тогда некоторая окрестность точки $M_{0}$ однозначно проецируется на касательную плоскость, проведённую в любой точке этой окрестности.
	\newline
	\newline
	\textit{Доказательство.}
	\newline
	Пусть $U$ некоторая окрестность точки $M_{0} | $ нормаль в любой точке этой окрестности составляет с нормалью точки $M_{0}$ угол, меньший $\frac{\pi}{4}$ \textit{(Такую окрестность выбрать можно в силу непрерывности векторного поля нормалей)}. И пусть $U$ однозначно проецируется на некоторый круг в одной из координатных плоскостей \textit{(Например, $O_{xy}$)}. Последнее условие гарантируется теоремой 13. Заметим, что нормали в любых двух точках образуют между собой угол, меньший $\frac{\pi}{2}$.
	\newline
	Пусть $U$ не обладает свойством, указанным в лемме 22, тогда для некоторой точки $M \in U$ можно найти такие точки $P, Q \in U$, что хорда $PQ$ параллельна нормали $\textbf{n}_{M}$ в точке $M$. Пусть плоскость $\Pi$ параллельна $O_{z}$ и $PQ \in \Pi$. Рассмотрим линию пересечения $U$ и $\Pi$. В силу выбора $U$ часть линии пересечения $PNQ$ представляет собой график дифференцируемой функции, заданной на отрезке, который является проекций $PQ$ на плоскость $O_{xy}$. По теореме Лагранжа для дифференцируемой функции, заданной на отрезке, касательная в некоторой точке $N$ этой части параллельна хорде $PQ$ и, следовательно, параллельна нормали $\textbf{n}_{M}$ в точке $M$. Но тогда нормаль в точке $N$, перпендикулярная касательной, упомянутой только что, образует с $\textbf{n}_{M}$ угол $\frac{\pi}{2}$, что приводит нас к противоречию. \textit{Лемма доказана.}
	\newline
	\newline
	\textit{\textbf{Лемма 23.}} Пусть $\Phi$ - гладкая, ограниченная полная поверхность без особых точек, тогда $\exists \delta \textgreater 0$, что любая часть $\Phi$, размеры которой меньше $\delta$, однозначно проецируется на касательную плоскость, проходящую через любую точку этой части.
	\newline
	\newline
	\textit{Доказательство.}
	\newline
	Пусть утверждение неверно: $\forall \delta_{n} = \frac{1}{n} \, (n = 1, 2, ...)$ можно указать часть $U_{n}$ поверхности $\Phi$, размеры которой меньше $\delta$ и которая не проецируется однозначно на касательную плоскость в некоторой своей точке. Выберем в каждой $U_{n}$ точку $M_{n}$, тогда получим последовательность точек $\{M_{n}\}$. Так как $\Phi$ ограниченная поверхность, последовательность $\{M_{n}\}$ ограничена, следовательно можно выделить подпоследовательность, сходящуюся к некоторой точке $M_{0}$ поверхности. Рассмотрим окрестность $M_{0}$, удовлетворяющую условиям леммы 22. При достаточно большом $n$ эта окрестность будет содержать в себе каждую часть $U_{n}$. Но тогда эта часть должна однозначно проецироваться на касательную плоскость в любой своей точке, что противоречит нашему предположению. Следовательно предположение неверно и \textit{лемма доказана}.
	\newline
	\newline
	\textit{\textbf{Лемма 24.}} Пусть $\Phi$ - гладкая, ограниченная полная поверхность без особых точек, тогда $\exists \delta \textgreater 0$, что любая часть $\Phi$, размеры которой меньше $\delta$, однозначно проецируется на одну из координатных плоскостей.
	\newline
	\newline
	\textit{Доказательство.}
	\newline
	Пусть утверждение неверно: $\forall \delta_{n} = \frac{1}{n} \, (n = 1, 2, ...)$ можно указать часть $U_{n}$ поверхности $\Phi$, размеры которой меньше $\delta$ и которая не проецируется однозначно на одну из координатных плоскостей. Выберем в каждой $U_{n}$ точку $M_{n}$, тогда получим последовательность точек $\{M_{n}\}$. Так как $\Phi$ ограниченная поверхность, последовательность $\{M_{n}\}$ ограничена, следовательно можно выделить подпоследовательность, сходящуюся к некоторой точке $M_{0}$ поверхности. Рассмотрим окрестность $M_{0}$, удовлетворяющую теореме 13 \textit{(см. замечание к теореме 13)}. При достаточно большом $n$ эта окрестность будет содержать в себе каждую часть $U_{n}$. Но тогда эта часть должна однозначно проецироваться на одну из координатных плоскостей в любой своей точке, что противоречит нашему предположению. Следовательно предположение неверно и \textit{лемма доказана}.
	\newline
	\newline
	\textit{\textbf{Лемма 25.}} Пусть $\Phi$ - гладкая, ограниченная, полная двусторонняя поверхность без особых точек, тогда 
	$\forall \varepsilon \textgreater 0 \ \exists \delta_{\varepsilon} \textgreater 0 \ | $ для $cos\gamma$, где $\gamma$ - угол между единичными векторами нормалей в любых двух точках произвольное части поверхности, размеры которой меньше $\delta$ справедливо представление $cos\gamma = 1 - \alpha_{\Phi}, \ 0 < \alpha_{\Phi} \textless \varepsilon$.
	\newline
	\newline	
	\textit{Доказательство.}
	\newline
	Рассмотрим на $\Phi$ непрерывное векторное поле единичных нормалей $\textbf{n}(M)$. $\Phi$ - двусторонняя поверхность, значит такое поле существует. $\Phi$ - ограниченная полная поверхность, поэтому представляет собой компактное множество. В результате получаем выполнение условий теоремы Кантора, отсюда $\textbf{n}(M)$ равномерно непрерывна на $\Phi$. Запишем определение равномерной непрерывности.
	\[ \forall \varepsilon  > 0 \quad \exists \delta_{\varepsilon} > 0 \quad | \quad \forall M_{1}, M_{2} \quad ||M_{2} - M_{1} || < \delta \Rightarrow ||\textbf{n}(M_{2}) - \textbf{n}(M_{1})|| < \sqrt{2\varepsilon} \]
	Заметим, что $||\textbf{n}(M_{2})||^2 = ||\textbf{n}(M_{1})||^2 = 1$ , $(\textbf{n}(M_{2}) , \textbf{n}(M_{1})) = cos\gamma$, тогда
	\newline
	\[\frac{1}{2}||\textbf{n}(M_{2}) - \textbf{n}(M_{1})||^2 = \frac{1}{2}(\textbf{n}(M_{2}) - \textbf{n}(M_{1}), \textbf{n}(M_{2}) - \textbf{n}(M_{1})) = \frac{1}{2}(||\textbf{n}(M_{2})||^2 - 2(\textbf{n}(M_{2}) , \textbf{n}(M_{1})) + ||\textbf{n}(M_{1})||^2) = \]
	\[ = \frac{1}{2}(2 - 2(\textbf{n}(M_{2}) , \textbf{n}(M_{1}))) = 1 - (\textbf{n}(M_{2}) , \textbf{n}(M_{1})) = 1 - cos\gamma\]
	Отсюда
	\[ cos\gamma = 1 - \frac{1}{2}||\textbf{n}(M_{2}) - \textbf{n}(M_{1})||^2 = 1 - \alpha_{\Phi}\]
	$\alpha_{\Phi} = \frac{1}{2}||\textbf{n}(M_{2}) - \textbf{n}(M_{1})||^2 < \frac{1}{2}2\varepsilon = \varepsilon$, что и требовалось доказать.
	
	\newpage
	\section{Площадь поверхности}
	Пусть $\Phi$ - ограниченная полная двусторонняя поверхность \textit{(+$\Phi$ - гладкая)}. Разобьём её кусочно-гладкими кривыми на конечное число частей $\Phi_{i}$, каждая из которых однозначно проецируется на касательную плоскость, проходящую через любую точку этой части \textit{(Лемма 23 гарантирует это)}. Пусть $\lambda$ - максимальный из размеров частей $\Phi_{i}$, $\sigma_{i}$ - площадь проекции $\Phi_{i}$ на касательную плоскость в некоторой точке $M_{i} \in \Phi_{i}$. Составим $\sum\limits_{i=1}^n \sigma_{i}$.
	\newline
	\textbf{Определение 26.} \textit{Пределом сумм} $\sum\limits_{i=1}^n \sigma_{i}$ назовём число $\sigma$, если
	\[ \forall\varepsilon > 0 \quad \exists\delta_{\varepsilon} > 0 \quad | \quad \forall T \quad \forall \Xi \quad\lambda < \delta \rightarrow |\sum\limits_{i=1}^n \sigma_{i} - \sigma| < \varepsilon \]
	\textbf{Определение 27.} Если предел таких сумм существует, то поверхность называется \textit{квадрируемой}, а число $\sigma$ - \textit{площадью повехности $\Phi$}.
	\newline
	\newline
	\textbf{\textit{Теорема 28.}} Гладкая ограниченная двусторонняя поверхность без особых точек квадрируема.
	\newline
	\newline
	\textit{Доказательство.}
	\newline
	Пусть на поверхности $\Phi$ может быть введена единая регулярная параметризация, тогда
	\[\forall M \in \Phi \quad \textbf{r}(M) = \textbf{r}(u,v) \in C^{1}(\Omega) \]
	$\Omega$ - ограниченная замкнутая область плоскости переменных $(u,v)$.
	\newline
	$\textbf{r}'_{u}, \textbf{r}'_{v}$ - непрерывные векторные функции, следовательно $|\left[ \textbf{r}'_{u}, \textbf{r}'_{v} \right] | = |\textbf{N}|$ - непрерывная функция. К тому же результат функции $|\textbf{N}|$ не зависит от выбора декартовой системы координат. Поэтому
	\[ \sigma = \iint\limits_{\Omega} | \left[ \textbf{r}'_{u}, \textbf{r}'_{v} \right] | dudv\]
	не зависит от выбора декартовой системы координат в пространстве.
	\newline
	\newline
	Докажем, что $\Phi$ квадрируема и её площадь равна $\sigma$. Пусть $\varepsilon$ - произвольное положительное число. Зафиксируем его и определим по этому $\varepsilon > 0$ число $\delta_{\varepsilon} > 0$, исходя из следующих требований:
	\newline
	1. Любая часть поверхности $\Phi$, размеры которой меньше $\delta$, однозначно проецируется на касательную плоскость в любой точке части. 
	\newline
	2. Косинус угла $\gamma$ между единичными векторами нормалей в любых двух точках части поверхности $\Phi$ может быть представлен в виде $cos\gamma = 1 - \alpha_{\Phi_{i}}$, $0 < \alpha_{\Phi_{i}} < \frac{\varepsilon}{\sigma}$ и $0 < \alpha_{\Phi_{i}} < 1$.
	\newline
	Выбор такого $\delta$ гарантируется леммами из предыдущего раздела.
	\newline
	\newline
	Рассмотрим произвольное разбиение $T$ поверхности $\Phi$ кусочно-гладкими кривыми на конечное число частей $\Phi_{i} \mid \lambda(T) < \delta$. Так как на $\Phi$ введена единая параметризация, то указанному разбиению $T$ отвечает разбиение $T'$ области $\Omega$ на конечное число $\Omega_{i}$. На каждой части $\Phi_{i}$ выберем произвольную точку $M_{i}$ и обозначим через $\sigma_{i}$ площадь проекции части $\Phi_{i}$ на касательную плоскость в точке $M_{i}$. Чтобы вычислить $\sigma_{i}$, поступим так. Выберем декартову систему координат так, чтобы её начало совпадало с $M_{i}$, ось $O_{z}$ была направлена по вектору нормали к поверхности в $M_{i}$, а оси $O_{y} , O_{z}$ были бы расположены в упомянутой ранее касательной плоскости. В указанной системе координат поверхность определяется параметрическими уравнениями $x = x(u,v), y = y(u,v), z = z(u,v)$, а вектор $\left[ \textbf{r}'_{u}, \textbf{r}'_{v} \right]$ имеет координаты $\{A, B, C\}$, где  
	
	\[ \left[ \textbf{r}'_{u}, \textbf{r}'_{v} \right] = 
	\begin{vmatrix}
		\textbf{i} & \textbf{j} & \textbf{k}\\
		x'_{u} & y'_{u} & z'_{u}\\
		x'_{v} & y'_{v} & z'_{v}
	\end{vmatrix} = 
	\begin{vmatrix}
		y'_{u} & z'_{u}\\
		y'_{v} & z'_{v}
	\end{vmatrix} \times \textbf{i} -
	\begin{vmatrix}
		x'_{u} & z'_{u}\\
		x'_{v} & z'_{v}
	\end{vmatrix} \times \textbf{j} +	
		\begin{vmatrix}
	x'_{u} & y'_{u}\\
	x'_{v} & y'_{v}
	\end{vmatrix} \times \textbf{k} =
	\]
	
	\begin{equation}\label{CoordVec} = 
	\begin{vmatrix}
		y'_{u} & z'_{u}\\
		y'_{v} & z'_{v}
	\end{vmatrix} \times \textbf{i} +
	\begin{vmatrix}
		z'_{u} & x'_{u}\\
		z'_{v} & x'_{v}
	\end{vmatrix} \times \textbf{j} +	
	\begin{vmatrix}
		x'_{u} & y'_{u}\\
		x'_{v} & y'_{v}
	\end{vmatrix} = A\textbf{i} + B\textbf{j} + C\textbf{k}
	\end{equation}
	\newline
	Для точек части $\Phi_{i}$, ввиду выбора $\delta$ и ориентации оси $O_{z}$, величина $C > 0$. Также $cos\gamma_{M} = \frac{C}{|\left[ \textbf{r}'_{u}, \textbf{r}'_{v} \right]|}$, где $\gamma_{M}$ - угол между нормалью в точке $M$ части $\Phi_{i}$ и осью $O_{z}$.
	\newline
	Ясно, что угол $\gamma_{M}$ является углом между нормалями в точках $M$ и $M_{i}$ части $\Phi_{i}$ \textit{(так как проводили ось $0_{z}$ через точку $M_{i}$)}, поэтому для него справедливо представление $cos\gamma = 1 - \alpha_{\Phi_{i}}$, $0 < \alpha_{\Phi_{i}} < \frac{\varepsilon}{\sigma}$ и $0 < \alpha_{\Phi_{i}} < 1$ \textit{(Лемма 25)}.
	\newline
	\newline
	Рассмотрим интеграл $\iint\limits_{\Omega_{i}} | \left[ \textbf{r}'_{u}, \textbf{r}'_{v} \right] | dudv$. Очевидно, что он не зависит от выбора декартовых координат в пространстве. Используем положительность $C$ и её выражение в виде определителя:
	\[\iint\limits_{\Omega_{i}} | \left[ \textbf{r}'_{u}, \textbf{r}'_{v} \right] | dudv = \iint\limits_{\Omega_{i}} \frac{| \left[ \textbf{r}'_{u}, \textbf{r}'_{v} \right] |}{
		\begin{vmatrix}
			x'_{u} & y'_{u}\\
			x'_{v} & y'_{v}
		\end{vmatrix}
	}
	\begin{vmatrix}
		x'_{u} & y'_{u}\\
		x'_{v} & y'_{v}
	\end{vmatrix} dudv\]
	Применим к интегралу в правой части равенства обобщённую теорему о среднем:
	\[ \bigg(\frac{| \left[ \textbf{r}'_{u}, \textbf{r}'_{v} \right] |}{
		\begin{vmatrix}
			x'_{u} & y'_{u}\\
			x'_{v} & y'_{v}
		\end{vmatrix}
	}\bigg)\bigg|_M
	\iint\limits_{\Omega_{i}} 
		\begin{vmatrix}
			x'_{u} & y'_{u}\\
			x'_{v} & y'_{v}
		\end{vmatrix} dudv
	=
	\bigg(\frac{| \left[ \textbf{r}'_{u}, \textbf{r}'_{v} \right] |}{
		\begin{vmatrix}
			x'_{u} & y'_{u}\\
			x'_{v} & y'_{v}
		\end{vmatrix}
	}\bigg)\bigg|_M
	\iint\limits_{\Omega_{i}} \bigg| \frac{D(x,y)}{D(u,v)} \bigg| dudv	
	\]
	$M$ - некоторая точка $\Phi_{i}$, отвечающая значениям параметров $(u,v)$.
	\newline
	Так как $cos\gamma_{M} = \frac{C}{|\left[ \textbf{r}'_{u}, \textbf{r}'_{v} \right]|}$, то 
	\[\frac{1}{cos\gamma_{M}} = \frac{|\left[ \textbf{r}'_{u}, \textbf{r}'_{v} \right]|}{C} = 
	\bigg( \frac{|\left[ \textbf{r}'_{u}, \textbf{r}'_{v} \right]|}
	{
		\begin{vmatrix}
			x'_{u} & y'_{u}\\
			x'_{v} & y'_{v}
		\end{vmatrix}
	}\bigg) \bigg|_M
	\]
	При этом
	\[
	\iint\limits_{\Omega_{i}} \bigg| \frac{D(x,y)}{D(u,v)} \bigg| dudv = \sigma_{i}
	\]
	Таким образом, используя представление $cos\gamma_{M} = 1 - \alpha_{\Phi_{i}}$, получаем, что
	\[
		\iint\limits_{\Omega_{i}} | \left[ \textbf{r}'_{u}, \textbf{r}'_{v} \right] | dudv =
		\bigg(\frac{| \left[ \textbf{r}'_{u}, \textbf{r}'_{v} \right] |}{
			\begin{vmatrix}
				x'_{u} & y'_{u}\\
				x'_{v} & y'_{v}
			\end{vmatrix}
		}\bigg)\bigg|_M
		\iint\limits_{\Omega_{i}} \bigg| \frac{D(x,y)}{D(u,v)} \bigg| dudv	=
		\frac{\sigma_{i}}{cos\gamma_{M}} = \frac{\sigma_{i}}{1 - \alpha_{\Phi_{i}}}
	\]
	Отсюда становится понятно, чему равно $\sigma_{i}$
	\[
		\sigma_{i} = (1 - \alpha_{\Phi_{i}}) \iint\limits_{\Omega_{i}} | \left[ \textbf{r}'_{u}, \textbf{r}'_{v} \right] | dudv = \iint\limits_{\Omega_{i}} | \left[ \textbf{r}'_{u}, \textbf{r}'_{v} \right] | dudv - \iint\limits_{\Omega_{i}} \alpha_{\Phi_{i}} | \left[ \textbf{r}'_{u}, \textbf{r}'_{v} \right] | dudv
	\]
	Просуммируем равенство по всем частям $\Phi_{i}$
	\[
		\sum\limits_{i=1}^n \sigma_{i} = \sum\limits_{i=1}^n \iint\limits_{\Omega_{i}} | \left[ \textbf{r}'_{u}, \textbf{r}'_{v} \right] | dudv - \sum\limits_{i=1}^n \iint\limits_{\Omega_{i}} \alpha_{\Phi_{i}} | \left[ \textbf{r}'_{u}, \textbf{r}'_{v} \right] | dudv =
		\sigma - \sum\limits_{i=1}^n \iint\limits_{\Omega_{i}} \alpha_{\Phi_{i}} | \left[ \textbf{r}'_{u}, \textbf{r}'_{v} \right] | dudv
	\]
	Осталось оценить нужную нам разность $|\sum\limits_{i=1}^n \sigma_{i} - \sigma| = |\sum\limits_{i=1}^n \iint\limits_{\Omega_{i}} \alpha_{\Phi_{i}} | \left[ \textbf{r}'_{u}, \textbf{r}'_{v} \right] | dudv|$
	\newline
	Используя свойства модуля и оценку $0 < \alpha_{\Phi_{i}} < \frac{\varepsilon}{\sigma}$, получим
	\[
		\bigg|\sum\limits_{i=1}^n \iint\limits_{\Omega_{i}} \alpha_{\Phi_{i}} | \left[ \textbf{r}'_{u}, \textbf{r}'_{v} \right] | dudv \bigg| < 
		\sum\limits_{i=1}^n \bigg| \iint\limits_{\Omega_{i}} \alpha_{\Phi_{i}} | \left[ \textbf{r}'_{u}, \textbf{r}'_{v} \right] | dudv \bigg| 
		< \sum\limits_{i=1}^n \iint\limits_{\Omega_{i}} |\alpha_{\Phi_{i}}||  \left[ \textbf{r}'_{u}, \textbf{r}'_{v} \right] | dudv = 
	\] 
	\[ = \sum\limits_{i=1}^n \iint\limits_{\Omega_{i}} \alpha_{\Phi_{i}}  |\left[ \textbf{r}'_{u}, \textbf{r}'_{v} \right] | dudv < \frac{\varepsilon}{\sigma} \sum\limits_{i=1}^n \iint\limits_{\Omega_{i}} | \left[ \textbf{r}'_{u}, \textbf{r}'_{v} \right] | dudv = \frac{\varepsilon}{\sigma}\times \sigma = \varepsilon
	\]
	Значит $|\sum\limits_{i=1}^n \sigma_{i} - \sigma| < \varepsilon$, поверхность $\Phi$ квадрируема и её площадь равна $\sigma$, \textit{что и требовалось доказать}.
	\newline 
	Был рассмотрен случай, когда на поверхности может быть задана единая параметризация. В общем случае поверхность может быть разбита на несколько частей, в каждой из которых может быть введена единая параметризация. После этого площадь поверхности можно определить как сумму площадей указанных частей.
	\newline
	\newline
	\textit{Замечания к теореме 28:}
	\newline
	\textit{1. Пусть поверхность $\Phi$ кусочно-гладкая, то есть составлена из конечного числа гладких ограниченных полных двусторонних поверхностей. Очевидно, что $\Phi$ квадрируема и её площадь может быть определена как сумма площадей составляющих её поверхностей.}
	\newline
	\textit{2.В процессе доказательства было установлено следуещее: если на поверхности $\Phi$ может быть введена единая параметризация и областью задания радиус-вектора $\textbf{r}(u,v)$ поверхности $\Phi$ является замкнутая ограниченная область $\Omega$ плоскости переменных $(u,v)$, то площадь поверхности может быть найдена по формуле}
	\begin{equation}\label{AreaSurf} 
		\sigma = \iint\limits_{\Omega} |\left[ \textbf{r}'_{u} , \textbf{r}'_{v} \right]| dudv
	\end{equation}
	\textit{Если $x = x(u,v), y = y(u,v), z = z(u,v) $ - параметрические функции поверхности, то вектор $\left[ \textbf{r}'_{u} , \textbf{r}'_{v} \right]$ имеет координаты $\left\{ A, B, C\right\}$, определяемые соотношениями} \eqref{CoordVec}. \textit{Поскольку $|\left[ \textbf{r}'_{u} , \textbf{r}'_{v} \right]| = \sqrt{A^2 + B^2 + C^2}$, то формула} \eqref{AreaSurf} \textit{может быть записана в виде:}
	\[\sigma = \iint\limits_{\Omega} \sqrt{A^2 + B^2 + C^2} dudv\]
	\textit{Рассмотрим выражение $A^2 + B^2 + C^2$, раскроем определители из формулы} \eqref{CoordVec}. \textit{Перегруппируем слагаемые, а затем прибавим и вычтем $(x'_{u}x'_{v})^2, (y'_{u}y'_{v})^2, (z'_{u}z'_{v})^2$}.
	\[A^2 + B^2 + C^2 = (y'_{u}z'_{v} - y'_{v}z'_{u})^2 + (z'_{u}x'_{v} - z'_{v}x'_{u})^2 + (x'_{u}y'_{v} - x'_{v}y'_{u})^2 = (y'_{u}z'_{v})^2 + (y'_{v}z'_{u})^2 - 2y'_{u}y'_{v}z'_{u}z'_{v} + \]
	\[ + (x'_{u}z'_{v})^2 + (x'_{v}z'_{u})^2 - 2x'_{u}x'_{v}z'_{u}z'_{v} + (x'_{u}y'_{v})^2 + (x'_{v}y'_{u})^2 - 2x'_{u}x'_{v}y'_{u}y'_{v} = (x'_{u}x'_{v})^2 + (x'_{u}y'_{v})^2 + (x'_{u}z'_{v})^2 +\]
	\[ + (x'_{v}y'_{u})^2 + (y'_{u}y'_{v})^2 + (y'_{u}z'_{v})^2 + (x'_{v}z'_{u})^2 + (y'_{v}z'_{u})^2 + (z'_{u}z'_{v})^2 - ((x'_{u}x'_{v})^2 + (y'_{u}y'_{v})^2 + (z'_{u}z'_{v})^2 - 2x'_{u}x'_{v}y'_{u}y'_{v} - \]
	\[- 2x'_{u}x'_{v}z'_{u}z'_{v} - 2y'_{u}y'_{v}z'_{u}z'_{v}) = x'^2_{u}(x'^2_{v} + y'^2_{v} + z'^2_{v}) + y'^2_{u}(x'^2_{v} + y'^2_{v} + z'^2_{v}) + z'^2_{u}(x'^2_{v} + y'^2_{v} + z'^2_{v}) -\]
	\[-(x'_{u}x'_{v} + y'_{u}y'_{v} + z'_{u}z'_{v})^2 = (x'^2_{u} + y'^2_{u} + z'^2_{u})(x'^2_{v} + y'^2_{v} + z'^2_{v}) -(x'_{u}x'_{v} + y'_{u}y'_{v} + z'_{u}z'_{v})^2 = EG - F^2\]
	\textit{Тогда} \eqref{AreaSurf} \textit{запишется в виде}
	\[\sigma = \iint\limits_{\Omega} \sqrt{EG - F^2} dudv\]
	\textit{где E, G, F - Гауссовы коэффициенты.}
	\newline
	\textit{Замечание 3. Площадь поверхности обладает свойством аддитивности: если поверхность $\Phi$ разбита кусочно-гладкой линией на неимеющие общих внутренних точек части $\Phi_{1}$ и $\Phi_{2}$, то площадь поверхности $\Phi$ равна сумме площадей частей $\Phi_{1}$ и $\Phi_{2}$. Это свойство вытекает из представления площади с помощью интеграла и аддитивного свойства интеграла.}
	
	\newpage
	\section{Поверхностные интегралы}
	Пусть $S$ - гладкая, ограниченная, полная, двусторонняя поверхность, на которой заданы скалярная функция $f(M)$ и векторная функция $\textbf{F}(M) = (P(M), Q(M), R(M))$. Через $\textbf{n}(M)$ обозначим непрерывное векторное поле единичных нормалей к $S$.
	\newline
	Разобьём поверхность $S$ кусочно-гладкими кривыми на части $S_{i}$ и на каждой из них выберем произвольно точку $M_{i}$. Пусть $\lambda$ - максимальный размер частей $S_{i}$, $\sigma_{i}$ - площадь $S_{i}$, $\alpha_{i}, \beta_{i}, \gamma_{i}$ - углы между вектором $\textbf{n}(M_{i})$ и осями $O_{x}, O_{y}, O_{z}$ соответственно.
	\newline
	Составим следующие четыре суммы:
	\begin{equation}
		I = \sum\limits_{i=1}^n f(M_{i})\sigma_{i} \quad
		I_{x} = \sum\limits_{i=1}^n P(M_{i})cos\alpha_{i}\sigma_{i} \quad
		I_{y} = \sum\limits_{i=1}^n Q(M_{i})cos\beta_{i}\sigma_{i} \quad
		I_{z} = \sum\limits_{i=1}^n R(M_{i})cos\gamma_{i}\sigma_{i} \quad
	\end{equation}
	\textbf{Определение 29.} \textit{Поверхностный интеграл 1го рода.}
	\newline
	Если \[\forall \varepsilon > 0 \quad \exists \delta_{\varepsilon} > 0 \quad|\quad \forall T \quad\forall \Xi \quad\lambda < \delta \rightarrow |I - J| < \varepsilon \]
	то $J$ называется \textit{поверхностным интегралом первого рода} от функции $f$ по поверхности $S$. 
	\newline
	Обозначение: $\iint\limits_{S} f(M)dS$
	\newline
	\textbf{Определение 30.} \textit{Поверхностный интеграл 2го рода.}
	\newline
	Если \[\forall \varepsilon > 0 \quad \exists \delta_{\varepsilon} > 0 \quad|\quad \forall T \quad\forall \Xi \quad\lambda < \delta \rightarrow |I_{x} + I_{y} + I_{z} - J| < \varepsilon \]
	то $J$ называется \textit{поверхностным интегралом второго рода} от векторной функции $\textbf{F}$ по поверхности $S$. 
	\newline
	Обозначение: $\iint\limits_{S} (\textbf{F}, \textbf{n}) dS = \iint\limits_{S} (P(M)cos\alpha + Q(M)cos\beta + R(M)cos\gamma) dS$
	\newline
	\newline
	Из определения поверхностного интеграла 1го рода следует его независимость от выбора ориентации векторного поля единичных нормалей к поверхности \textit{(иными словами от выбора стороны поверхности, по которой интегрируем)}.
	\newline
	В свою очередь поверхностный интеграл 2го рода зависит от выбора стороны поверхности. При изменении ориентации векторного поля единичных нормалей на противоположную поверхностный интеграл меняет свой знак на противоположный. Это происходит из-за наличия в суммах $I_{x}, I_{y}, I_{z}$ косинусов углов, который составляет нормаль с осями координат, при изменении ориентации векторного поля нормалей косинусы меняют свой знак на противоположный. После выбора определённой стороны поверхности поверхностный интегралы 2го рода могут рассматриваться как 1го рода от функции $P(M)cos\alpha(M) + Q(M)cos\beta(M) + R(M)cos\gamma(M)$.
	\newline
	\newline
	Пусть $S$ - гладкая, ограниченная, полная, двусторонняя поверхность. Выберем на $S$ определённую сторону. Согласно предыдущим замечаниям поверхностные интегралы второго рода можно рассматривать как интегралы первого рода, поэтому достаточные условия существования будут формулироваться для интегралов первого рода. 
	\newline
	\newline
	\textbf{\textit{Теорема 31.}} Пусть на $S$ можно ввести единую параметризацию посредством функций \[ x = x(u,v) , y = y(u,v), z = z(u,v)\] заданных в ограниченной и замкнутой области $\Omega$ плоскости переменных $(u,v)$  и принадлежащих классу $C^{1}$. Если $f(M) = f(x,y,z)$ непрерывна на $S$, то поверхностный интеграл первого рода от функции $f$ по поверхности $S$ существует и может быть найден по формуле:
	\begin{equation}\label{IntForm}
		I = \iint\limits_{S} f(M)dS = \iint\limits_{\Omega} f(x(u,v), y(u,v), z(u,v)) \sqrt{EG - F^2} dudv
	\end{equation}
	\newline
	\textit{Доказательство.}
	\newline
	Хотим, чтобы
	\[ \forall \varepsilon > 0 \quad \exists \delta_{\varepsilon} > 0 \quad|\quad \forall T \quad\forall \Xi \quad\lambda < \delta \rightarrow |\sum\limits_{i=1}^n f(M_{i})\sigma_{i} - \iint\limits_{\Omega} f(x(u,v), y(u,v), z(u,v)) \sqrt{EG - F^2} dudv| < \varepsilon \]
	Для этого выберем любое положительное число $\varepsilon$ и зафиксируем его. Найдём по нему число $\delta^{*} > 0$ такое, что
	\newline
	1. $\forall N_{i} = (u_{i}, v_{i}) \in \Omega, \forall \tilde N_{i} = (\tilde u_{i}, \tilde v_{i}) \in \Omega \quad | \quad ||\tilde N_{i} - N_{i}|| < \delta^{*}$ выполнялось: 
	\[ |\sqrt{E(\tilde N_{i})G(\tilde N_{i}) - F^2(\tilde N_{i})} - \sqrt{E(N_{i})G(N_{i}) - F^2(N_{i})}| < \frac{\varepsilon}{2AP}\]
	где $A > \sup\limits_{M \in S}|f(M)|$ и $P = |\Omega|$.
	\newline
	2. Для любого разбиения $\Omega$ кусочно-гладкими кривыми на конечное число $\Omega_{i}$ таких, что $|\Omega_{i}| < \delta^{*}$, и $\forall N_{i} = (u_{i}, v_{i}) \in \Omega_{i}$ выполнялось:
	\[ |   \sum\limits_{i=1}^n f(x((u_{i}, v_{i})), y((u_{i}, v_{i})), z((u_{i}, v_{i}))\sqrt{E(u_{i}, v_{i})G(u_{i}, v_{i}) - F(u_{i}, v_{i})^2}|\Omega{i}| - \]
	\[ - \iint\limits_{\Omega} f(x(u,v), y(u,v), z(u,v)) \sqrt{EG - F^2} dudv | < \frac{\varepsilon}{2}\]
	Почему такое $\delta^{*}$ можно выбрать?
	\newline
	Обратим внимание на первое условие. Так как $E, G, F$ это скалярные произведения радиус-векторов $\textbf{r}'_{u}, \textbf{r}'_{v}$, координаты которых соответственно являются непрерывными функциями $(x'_{u}, y'_{u}, z'_{u})$ и $(x'_{v}, y'_{v}, z'_{v})$ в силу того, что $x(u,v), y(u,v), z(u,v) \in C^{1}(\Omega)$, то $E, F, G$ суть непрерывные функции. Следовательно $\sqrt{EG - F^2}$ - непрерывна как суперпозиция. $\Omega$ - ограниченная и замкнутая область, следовательно для функции $\sqrt{EG - F^2}$ действует теорема Кантора, условие которой и записано в пункте 1.
	\newline
	Второе условие есть ни что иное, как определение двойного интеграла для непрерывной функции $f$, которая непрерывна как суперпозиция, по области $\Omega$. А как известно, любая непрерывная функция интегрируема.
	\newline
	\newline
	Теперь по $\delta^{*} > 0$ определим $\delta > 0$  так, чтобы любому разбиению $S$ кусочно-гладкими кривыми на конечное число $S_{i} \quad | \quad |S_{i}| < \delta$ отвечало бы разбиение $\Omega$ кусочно-гладкими кривыми на конечное число $\Omega_{i} \quad | \quad |\Omega_{i}| < \delta^{*}$. Возможность такого $\delta$ гарантируется тем, что $S$ есть гомеоморфное отображение $\Omega$, поэтому найти такие разбиения можно, при этом если $\lambda(T') \rightarrow 0$, где $T'$ - разбиение $S$, то и $\lambda(T'') \rightarrow 0$, где $T''$ - разбиение $\Omega$.
	\newline
	\newline
	Рассмотрим разбиение $T$ поверхности $S$ кусочно-гладкими кривыми на конечное число $S_{i}$ таких, что $\lambda(T) < \delta$, где $\delta$ выбрано по $\delta^{*}$ способом, указанным ранее. Составим теперь интегральную сумму:
	\[ \sum\limits_{i=1}^n f(M_{i})\sigma_{i} = \sum\limits_{i=1}^n f(x(u_{i}, v_{i}),y(u_{i}, v_{i}),z(u_{i}, v_{i}))\iint\limits_{\Omega_{i}} \sqrt{EG - F^2} dudv \]
	учитывая, что $M_{i} = (x(u_{i}, v_{i}),y(u_{i}, v_{i}),z(u_{i}, v_{i}))$ и $\sigma_{i} = \iint\limits_{\Omega_{i}} \sqrt{EG - F^2} dudv$. Далее применим к интегралу $\iint\limits_{\Omega_{i}} \sqrt{EG - F^2} dudv$ теорему о среднем:
	\[ \sum\limits_{i=1}^n f(x(u_{i}, v_{i}),y(u_{i}, v_{i}),z(u_{i}, v_{i}))\sqrt{E(\tilde u_{i}, \tilde v_{i}) G(\tilde u_{i}, \tilde v_{i}) - F(\tilde u_{i}, \tilde v_{i})^2}|\Omega_{i}| \]
	\newline
	Для интеграла $\iint\limits_{\Omega} f(x(u,v), y(u,v), z(u,v)) \sqrt{EG - F^2} dudv$ составим интегральную сумму:
	\[ \sum\limits_{i=1}^n f(x(u_{i},v_{i}), y(u_{i},v_{i}), z(u_{i},v_{i})) \sqrt{E(u_{i},v_{i})G(u_{i},v_{i}) - F^2(u_{i},v_{i})}|\Omega_{i}| = I \]
	\newline
	Сравним следующую разность:
	\[ | \sum\limits_{i=1}^n f(x(u_{i}, v_{i}),y(u_{i}, v_{i}),z(u_{i}, v_{i}))\sqrt{E(\tilde u_{i}, \tilde v_{i}) G(\tilde u_{i}, \tilde v_{i}) - F(\tilde u_{i}, \tilde v_{i})^2}|\Omega_{i}|  - \]
	\[ - \iint\limits_{\Omega} f(x(u,v), y(u,v), z(u,v)) \sqrt{EG - F^2} dudv |\]
	Вычтем и прибавим $I$:
	\[ | \sum\limits_{i=1}^n f(x(u_{i}, v_{i}),y(u_{i}, v_{i}),z(u_{i}, v_{i}))\sqrt{E(\tilde u_{i}, \tilde v_{i}) G(\tilde u_{i}, \tilde v_{i}) - F(\tilde u_{i}, \tilde v_{i})^2}|\Omega_{i}|  - \]
	\[ - \sum\limits_{i=1}^n f(x(u_{i},v_{i}), y(u_{i},v_{i}), z(u_{i},v_{i})) \sqrt{E(u_{i},v_{i})G(u_{i},v_{i}) - F^2(u_{i},v_{i})}|\Omega_{i}| + \]
	\[ + \sum\limits_{i=1}^n f(x(u_{i},v_{i}), y(u_{i},v_{i}), z(u_{i},v_{i})) \sqrt{E(u_{i},v_{i})G(u_{i},v_{i}) - F^2(u_{i},v_{i})}|\Omega_{i}| - \]
	\[ - \iint\limits_{\Omega} f(x(u,v), y(u,v), z(u,v)) \sqrt{EG - F^2} dudv | <\]
	Воспользуемся свойствами модуля, чтобы оценить разность:
	\[ < | \sum\limits_{i=1}^n f(x(u_{i}, v_{i}),y(u_{i}, v_{i}),z(u_{i}, v_{i}))\sqrt{E(\tilde u_{i}, \tilde v_{i}) G(\tilde u_{i}, \tilde v_{i}) - F(\tilde u_{i}, \tilde v_{i})^2}|\Omega_{i}|  - \]
	\[ - \sum\limits_{i=1}^n f(x(u_{i},v_{i}), y(u_{i},v_{i}), z(u_{i},v_{i})) \sqrt{E(u_{i},v_{i})G(u_{i},v_{i}) - F^2(u_{i},v_{i})}|\Omega_{i}| |+ \]
	\[ + | \sum\limits_{i=1}^n f(x(u_{i},v_{i}), y(u_{i},v_{i}), z(u_{i},v_{i})) \sqrt{E(u_{i},v_{i})G(u_{i},v_{i}) - F^2(u_{i},v_{i})}|\Omega_{i}| - \]
	\[ - \iint\limits_{\Omega} f(x(u,v), y(u,v), z(u,v)) \sqrt{EG - F^2} dudv | <\]
	Очевидно, что для последнего модуля суммы мы уже получили оценку в пункте 2, в первом модуле суммы занесём всё под один знак суммы и вынесем общий множитель:
	\[ < | \sum\limits_{i=1}^n f(x(u_{i}, v_{i}),y(u_{i}, v_{i}),z(u_{i}, v_{i}))|\Omega_{i}| * \]
	\[ *(\sqrt{E(\tilde u_{i}, \tilde v_{i}) G(\tilde u_{i}, \tilde v_{i}) - F(\tilde u_{i}, \tilde v_{i})^2}  - \sqrt{E(u_{i},v_{i})G(u_{i},v_{i}) - F^2(u_{i},v_{i})})|+ \frac{\varepsilon}{2} <\]
	Воспользуемся свойством модуля суммы и модуля произведения:
	\[ < \sum\limits_{i=1}^n |f(x(u_{i}, v_{i}),y(u_{i}, v_{i}),z(u_{i}, v_{i}))|*|\Omega_{i}| * \]
	\[ *|\sqrt{E(\tilde u_{i}, \tilde v_{i}) G(\tilde u_{i}, \tilde v_{i}) - F(\tilde u_{i}, \tilde v_{i})^2}  - \sqrt{E(u_{i},v_{i})G(u_{i},v_{i}) - F^2(u_{i},v_{i})})|+ \frac{\varepsilon}{2}\]
	Для множителей $|\sqrt{E(\tilde u_{i}, \tilde v_{i}) G(\tilde u_{i}, \tilde v_{i}) - F(\tilde u_{i}, \tilde v_{i})^2}  - \sqrt{E(u_{i},v_{i})G(u_{i},v_{i}) - F^2(u_{i},v_{i})}|$ имеем оценку, полученную в пункте 1:
	\[ < \frac{\varepsilon}{2AP}\sum\limits_{i=1}^n |f(x(u_{i}, v_{i}),y(u_{i}, v_{i}),z(u_{i}, v_{i}))|*|\Omega_{i}| + \frac{\varepsilon}{2} <\]
	Так как $f(x(u_{i}, v_{i}),y(u_{i}, v_{i}),z(u_{i}, v_{i}))$ суть непрерывная функция от $(u,v)$, где $(u,v) \in \Omega$ - ограниченное и замкнутое множество, то применив к ней теорему Вейерштрасса, получим оценку $|f(x(u_{i}, v_{i}),y(u_{i}, v_{i}),z(u_{i}, v_{i}))| \leq \sup\limits_{M \in S}|f(M)| < A$, где $M = (x(u,v), y(u,v), z(u,v))$.
	\newline 
	Отсюда, зная, что $P = |\Omega|$, получим, что:
	\[ < \frac{\varepsilon}{2AP}A\sum\limits_{i=1}^n |\Omega_{i}| + \frac{\varepsilon}{2} = \frac{\varepsilon}{2AP}AP + \frac{\varepsilon}{2} = \varepsilon\] 
	\textit{что и требовалось доказать}.
	\newline
	\newline
	\textit{Замечания к теореме 31.}
	\newline
	\textit{1. Для поверхностных интегралов второго рода после выбора определённой стороны поверхности $S$ применима формула \eqref{IntForm}.}
	\newline
	\textit{2. Пусть поверхность $S$ является графиком функции $z = z(x,y) \in C^{1}(D)$. Выберем на поверхности такую сторону, чтобы вектор нормали поверхности составлял с осью $O_{z}$ острый угол $\gamma$. В таком случае $cos\gamma = \frac{1}{\sqrt{1 + p^{2} + q^{2}}}$, где $p = \frac{\partial z}{\partial x}, q = \frac{\partial z}{\partial y}$. Пусть на $S$ задана непрерывная векторная функция $(0,0,R(x,y,z))$. Возьмём в качестве параметров переменные $x, y$, тогда параметрическое задание поверхности $S$ будет иметь вид: $x = x, \; y = y, \; z = z(x,y)$. Используя формулу \eqref{IntForm}, запишем:}
	\[ \iint\limits_{S} (P(x,y,z)cos\alpha + Q(x,y,z)cos\beta + R(x,y,z)cos\gamma) dS = \iint\limits_{S} R(x,y,z) \frac{1}{\sqrt{1 + p^{2} + q^{2}}}dS =\]
	\[ = \iint\limits_{D} R(x,y,z(x,y)) \frac{1}{\sqrt{1 + p^{2} + q^{2}}} \sqrt{1 + p^{2} + q^{2}}dxdy = \iint\limits_{D} R(x,y,z(x,y)) dxdy\]
	\textit{Таким образом разъяснили обозначение поверхностного интеграла второго рода в виде:}
	\[ \iint\limits_{S} R(x,y,z) dxdy\]
	\textit{Такое обозначение может использоваться даже если $S$ не является графиком функции $z = z(x,y)$}
	\newline 
	Из замечания 2 получили $\iint\limits_{S} (P(M)cos\alpha + Q(M)cos\beta + R(M)cos\gamma) dS = \iint\limits_{S} P(M)dydz + Q(M)dxdz + R(M)dxdy$.
	\newline
	\textit{3. Понятия поверхностных интегралов и доказанная для них теорему естественно распространить на случай кусочно-гладкой поверхности $S$.}
	\newline
	\newline
	Обратимся к интегралу второго рода: $\iint\limits_{S} (\textbf{F}, \textbf{n}) dS = \iint\limits_{S} (P(M)cos\alpha + Q(M)cos\beta + R(M)cos\gamma) dS$.  $(\textbf{F}, \textbf{n})$ - непрерывная скалярная функция, заданная на $S$, не зависящая от выбора декартовой системы координат. Интеграл в правой части представляет собой сумму трёх поверхностных интегралов второго рода от скалярных функций и называется \textit{общим поверхностным интегралом второго рода}. Определить же поверхностный интеграл второго рода от скалярной функции просто, заменив лишь в суммах $I_{x}, I_{y}, I_{z}$ функции $P, Q, R$ на одну функцию, допустим $f$, определённую на поверхности $S$. Тогда если перейти к пределу интегральной суммы, к примеру $I_{x}$, при $\lambda(T) \rightarrow 0$ получим $\iint\limits_{S}f(x,y,z)cos\alpha dS$.
	\newline
	\textit{Замечание 1. $\iint\limits_{S} (\textbf{F}, \textbf{n}) dS$ - инвариантный вид общего поверхностного интеграла второго рода.}
	\newline
	\textit{Замечание 2. $\iint\limits_{S} (\textbf{F}, \textbf{n}) dS$ численно равен величине называемой в физике потоком вектора $\textbf{F}(M)$ через поверхность $S$.}

	\newpage
	\section{Формула Грина}
	Рассмотрим на плоскости $(x,y)$ криволинейную трапецию $D$, ограниченную контуром $L$.
	\newline
	$D = \{(x,y) \; | \; a \leq x \leq b, \; y_{1}(x) \leq y \leq y_{2}(x)\}$
	\newline
	$L_{1} = y_{2}(x), a \leq x \leq b, \; L_{2} = a,\; L_{3} = y_{1}(x), a \leq x \leq b,\; L_{4} = b$
	\newline
	Пусть в $D$ задана гладкая функция $P(x,y)$, можем вычислить двойной интеграл по области $D$ от частной производной по $y$ от этой функции.
	\[ \iint\limits_{D} \frac{\partial P}{\partial y} dxdy = \int\limits_a^b dx \int\limits_{y_{1}(x)} ^ {y_{2}(x)} \frac{\partial P}{\partial y}dy = \int\limits_a^b (P(x, y_{2}(x)) - P(x, y_{1}(x))) dx = \int\limits_a^b P(x, y_{2}(x)) dx - \int\limits_a^b P(x, y_{1}(x)) dx\]
	Теперь очевидно, что каждый из интегралов в правой части можно заменить криволинейным интегралом.
	\[ \int\limits_{L_{1}^{-}} P(x,y) dx - \int\limits_{L_{3}^{+}} P(x,y) dx = \int\limits_{L_{1}^{-}} P(x,y) dx + \int\limits_{L_{3}^{-}} P(x,y) dx\]
	Хотим интеграл по всему контуру $L$. Для этого прибавим криволинейные интегралы по $L_{2}$ и $L_{4}$.
	\[ \int\limits_{L_{2}^{-}} P(x,y) dx = \int\limits_a^a P(x,y) dx = 0\]
	\[ \int\limits_{L_{4}^{-}} P(x,y) dx = \int\limits_b^b P(x,y) dx = 0\]
	\[  \int\limits_{L_{1}^{-}} P(x,y) dx + \int\limits_{L_{2}^{-}} P(x,y) dx + \int\limits_{L_{3}^{-}} P(x,y) dx + \int\limits_{L_{4}^{-}} P(x,y) dx = \int\limits_{L^{-}} P(x,y) dx = -\int\limits_{L^{+}} P(x,y) dx \]
	Таким образом получили:
	\[ \iint\limits_{D} \frac{\partial P}{\partial y} dxdy = -\int\limits_{L^{+}} P(x,y) dx \]
	Формула доказана для более простого вида областей, но доказано, что она остаётся справедливой и для областей более сложного вида. Чтобы доказать это, достаточно предположить, что фигуру $D$ прямыми, параллельными $O_{y}$, можно разложить на конечно число упомянутых трапеций. Написав для каждой из них полученную формулу, сложим все равенства. Тогда слева получим двойной интеграл по области $D$, справа - сумма интегралов, взятых по всем частичным контурам, которая приводится к интегралу по всему контуру $L$, так как интегралы по вспомогательным отрезках, параллельным $O_{y}$, равны нулю.
	\newline
	\newline
	Аналогичным образом устанавливается формула и для случая, когда область $D$ представляется собой так называемую $x$-трапецеивидную область. В предыдущем случае $D$ называлась $y$-трапецеивидной ($Q$ - гладкая в $D$). И аналогичным образом обобщается на случай более сложной области.
	\[ \iint\limits_{D} \frac{\partial Q}{\partial x} dxdy = \int\limits_{L^{+}} Q(x,y) dy \]
	Если область $D$ может быть разбита указанным ранее способом на конечное число $x$-трапецеивидных и $y$-трапецеивидных областей, то для неё справедливы все полученные ранее формулы, в предположении гладкости функций $P$ и $Q$.
	Вычтем полученные результаты и выведем \textit{формулу Грина}.
	\[ \int\limits_{L^{+}} P(x,y)dx + Q(x,y) dy = \iint\limits_{D} (\frac{\partial Q}{\partial x} - \frac{\partial P}{\partial y} ) dxdy\]
	\textit{Формула так же имеет место быть и для области, ограниченной кусочно-гладкими контуроми.}
\end{document} 